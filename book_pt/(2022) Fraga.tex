\documentclass{book}
\usepackage[brazilian]{babel}
\usepackage[utf8]{inputenc}
\usepackage[T1]{fontenc}
\usepackage{multirow}

\usepackage{amsmath}
\usepackage{graphicx}
\graphicspath{ {Figures/} }
\usepackage{float}
\usepackage[dvipsnames]{xcolor}
\usepackage{tcolorbox}

\usepackage{natbib,stfloats}

\tcbuselibrary{skins,breakable}
\usetikzlibrary{shadings,shadows}

\usepackage{tikz}
\usetikzlibrary{snakes}
\usetikzlibrary{patterns}

\usepackage{fancyhdr}
\pagestyle{fancy}

\fancyhead{}
\fancyhead[RO]{\textsl{\rightmark}} 
\fancyhead[LE]{\textsl{\leftmark}} 

\fancyfoot{} 
\fancyfoot[L]{T.B. Fraga (2022). Gestão de Demanda e Controle de Estoque.}
\fancyfoot[R]{\thepage}

\definecolor{cambridgeblue}{rgb}{0.64, 0.76, 0.68}
\definecolor{camel}{rgb}{0.76, 0.6, 0.42}
\definecolor{camouflagegreen}{rgb}{0.47, 0.53, 0.42}
\definecolor{desertsand}{rgb}{0.93, 0.79, 0.69}

\newenvironment{exerciseblock}[1]{%
    \tcolorbox[beamer,%
    noparskip,breakable,
    colback=camel!20,colframe=camouflagegreen,%
    colbacklower=camouflagegreen!75!camel!20,%
    title=#1]}%
    {\endtcolorbox}

\newenvironment{exempleblock}[1]{%
    \tcolorbox[beamer,%
    noparskip,breakable,
    colback=camel!20,colframe=Brown!70,%
    colbacklower=camouflagegreen!75!camel!20,%
    title=#1]}%
    {\endtcolorbox}

\newenvironment{algorithmblock}[1]{%
    \tcolorbox[beamer,%
    noparskip,breakable,
    colback=camel!20,colframe=Brown!40,%
    colbacklower=camouflagegreen!75!camel!20,%
    title=#1]}%
    {\endtcolorbox}

\newenvironment{dedication}
  {\clearpage           % we want a new page
   \thispagestyle{empty}% no header and footer
   \vspace*{\stretch{1}}% some space at the top 
   \itshape             % the text is in italics
   \raggedleft          % flush to the right margin
  }
  {\par % end the paragraph
   \vspace{\stretch{3}} % space at bottom is three times that at the top
   \clearpage           % finish off the page
  }

% para construção de diagramas

\tikzstyle{place}=[circle,draw=camouflagegreen!80,fill=desertsand!20,thick]
\tikzstyle{Ret}=[rectangle,draw=camouflagegreen!80,fill=desertsand!20,thick]
\tikzstyle{RoundRet}=[rectangle,draw=camouflagegreen!80,fill=desertsand!20,thick, rounded corners]

\begin{document}

Livro em processo de elaboração. \\

Título: Gestão de Demanda e Controle de Estoque.\\

Autora: Tatiana Balbi Fraga. \\

Profa. do Núcleo de Tecnologia da UFPE. \\

Caso tenha interesse em participar da elaboração deste livro, enviar e-mail para: tatiana.balbi@ufpe.br.\\

O mesmo endereço de e-mail poderá ser utilizado para críticas e sugestões. Desde já agradeço por qualquer contribuição. \\

O conteúdo apresentado neste livro estará sendo modificado ao longo de sua elaboração, caso o mesmo seja consultado para elaboração de algum trabalho, favor citar como referência o título, a autora, a data da consulta e o endereço da versão consultada no github. \\

\chapter{Classificação multicritério para itens de inventário}

No intuito de oferecer um pronto atendimento aos clientes, grande parte das empresas estocam alguns produtos que comercializam. Contudo, o estoque apresenta uma série de preocupações tais como a necessidade de deixar um grande capital parado e o uso de amplo espaço para estocagem. Assim torna-se necessária uma correta identificação dos níveis adequados de estoque para os diferentes produtos através de estudos que buscam um ponto de equilíbrio entre as reais necessidades de estocagem e os custos envolvidos. Entre os esforço necessários para a definição de um ponto de equilíbrio podemos citar os estudos de demanda e a busca pela otimização da utilização do espaço disponível para estocagem. \\

Como muitas organizações comercializam centenas ou até mesmo milhares de itens de inventário, muitas vezes determinar o nível adequado de estoque para cada item se torna excessivamente custoso ou até mesmo impraticável. Portanto uma prática comum consiste em classificar os diversos itens dando a atenção necessária apenas aos itens considerados mais importantes. Entre os métodos de classificação mais conhecidos está o método de Análise ABC desenvolvido com base na observação de Pareto, que posteriormente foi generalizada como regra de Pareto ou regra 80-20. De acordo com essa regra, aproximadamente 20\% das causas é responsável por 80\% dos efeitos.   \\

De acordo com a classificação ABC, os itens de inventário devem ser distribuídos em três grupos - A, B e C - com base em algum critério conhecido. Os itens classificados como A são aqueles mais importantes, os itens B são menos importantes, e os itens C não apresentam importância significativa. Para classificação dos itens é necessário inicialmente listar os itens em ordem decrescente, de acordo com sua contribuição para o critério observado. Posteriormente, a distribuição dos itens entre as três categorias pode ser feita de acordo com a Tabela \ref{tab:ABCRules} apresentada a seguir. \\

\begin{table}[h]
\begin{center}
\begin{tabular}[c]{c c c}
\cline {1-3}
classe & percentual de itens & contribuição percentual \\ \cline {1-3}
A & 20\% &  80\%   \\ 
B & 50\% &  15\%   \\
C & 30\% &  5\% \\ \cline {1-3}
\end{tabular}
\label{tab:ABCRules}
\caption{Regra sugerida para classificação ABC.}
\end{center}
\end{table}

Obeserve que nem sempre será possível encontrar a proporção exata apresentada na Tabela \ref{tab:ABCRules}. Uma forma de contornar essa dificuldade consite em fixar aproximadamente um dos critérios encontrando o valor para o próximo. Por exemplo, em determinada análise pode ser observado que 10\% dos itens são responsáveis por 79,5\% dos resultados. 


\chapter{Identificação de padrões de demanda}

A previsão de demanda é essencial para o bom planejamento em qualquer empresa. Através de uma boa previsão é possível, entre outras coisas, controlar melhor os níveis de estoque, reduzindo custos e oferecendo um melhor nível de serviço aos clientes. 

Conforme mostra \cite{MakridakisHibon2000}, a literatura apresenta uma grande variedade de metodologias que podem ser utilizadas para previões de demanda, sendo que a performance dos distintos modelos de previsão varia de acordo a natureza dos dados e um modelo que gera bons resultados para determinada classe de itens de uma empresa pode gerar previsões ruins para outros itens dessa mesma empresa.

Uma estratégia natural utilizada para identificar o modelo de previsão adequado pra cada item consiste em comparar a performance dos distintos modelos candidatos utilizando dados históricos de vendas do item \citep{UlrichEtAl2022}. Contudo, como geralmente as empresas produzem e/ou comercializam uma grande variedade de itens, essa estratégia acaba se tornando um esforço considerável. 

De acordo com \cite{UlrichEtAl2022}, uma opção viável consiste em agrupar os itens de acordo com seus padrões de demanda, para posteriormente identificar o modelo de previsão adequado pra cada grupo e não mais para cada item individual. Esse capítulo apresenta, de forma bem detalhada, algumas das principais metodologias utilizadas para identificação de padrões de demanda e agrupamento dos itens de acordo com os padrões identificados.

\section{Classificação ABC}

O princípio de Pareto atenta para o fato de que, em muitas circunstâncias, um pequeno número de fatores são responsáveis pela maior parte impácto nos resultados \citep{HarveySotardi2018}. Este princípio é geralmente descrito como a regra 80-20, ou seja, quando 20\% dos fatores são responsáveis por 80\% do resultado. 



\section{Padrões de demanda}

'Os padrões de demanda são resultados da variação da demanda com o tempo, ou seja, do crescimento ou declínio de taxas de demanda, sazonalidades e flutuações gerais causadas por diversos fatores' (\cite{Ballou2001} apud \cite{WernerEtAl2006}). 

De acordo com \cite{Ballou2006}, quando a demanda apresenta comportamento regular, os padrões de demanda podem ser divididos em compo­nentes de tendência, sazonais ou aleatórios. Já nos casos em que a demanda de determinados itens é inter­mitente, em função do baixo volume geral e da incerte­za quanto a quando e em que nível essa demanda ocor­rerá, a série de tempo é chamada de incerta, ou irregu­lar.

\cite{BoylanEtAl2008} distribuem os padrões de demanda entre normais, onde a demanda pode ser representada por uma distribuição normal, e não normais, no caso em que isso não é possível. De acordo com os autores, os padrões de demanda não normais podem ser classificados da seguinte forma:

\begin{itemize}
  \item um item de \emph{demanda intermitente (intermittent)} é um item com ocorrências de demanda pouco frequêntes;
  \item um item de \emph{demanda de movimento lento (slow moving)} é um item cuja demanda média por período é baixa. Isso pode ser devido a ocorrências de demanda pouco frequentes, tamanhos médios de demanda baixos ou ambos;
  \item um item de \emph{demanda errática (erratic)} é um item cujo tamanho de demanda é altamente variável;
  \item um item de \emph{demanda esporádica (lumpy)} é um item intermitente para o qual a demanda, quando ocorre, é altamente variável.; e
  \item um item de \emph{demanda agregada (clumped)} é um item intermitente para o qual a demanda, quando ocorre, é constante (ou quase constante).
\end{itemize}

Apesar da importância da identificação dos padrões de demanda para identificação dos métodos adequados de previsão, poucos autores tratam deste assunto e poucas técnicas são apresentadas na literatura para essa finalidade.

\cite{BusingerRead1999} aplicam um sistema de agrupamento de itens utilizando um diagrama de plotagem em estrela considerando oito características dos dados de séries temporais: coeficiente de variação, número de zeros, tendência, picos (outliers), sazonalidade, corridas, assimetria e autocorrelação.

O coeficiente de variação $(CV)$ informa a variabilidade em relação à média. Essa medida adimensional informa o nível de dispersão: quanto mais alto for o $(CV)$, mais alta é a dispersão:

\begin{equation}
CV = \frac{s}{\bar{i}}
\end{equation}

onde: 

$\bar{i}$ é a média dos valores considerados, e

$s$ é o desvio padrão 

\begin{equation}
s = \sqrt{ \frac{1}{N} \sum_{i=1}^{N}{(i-\bar{i})^2}}
\end{equation}

O número de zeros $(NZ)$ é uma medida que indica o número de períodos com demanda zero (nula) dentro de determinado intervalo de tempo. 

\begin{equation}
NZ = \sum_{i=1}^{N}{I(y_i=0)}
\end{equation}

Onde $I(A)=1$, se $A$ é verdadeiro, e $I(A)=0$, se $A$ é falso.

Essa medida é, em alguns casos, associada à intermitência. Sendo $BP$ um valor de corte, se $(NZ\geq BP)$ então a demanda é considerada intermitente \citep{BoylanEtAl2008}.

A tendência $(T)$ apresenta um padrão de variação suave e temporário na demanda. Para cáculo da tendência, \cite{BusingerRead1999} dividem os dados avaliados em terços e calculam a tendência usando a seguinte expressão:

\begin{equation}
T = \frac{(Y_U - Y_L)}{(Y_{(\frac{5}{6})} - Y_{(\frac{1}{6})})}
\end{equation}

Onde $Y_U$ e $Y_L$ representam as medianas dos terços extremos, sendo $L$ o terço inferior, e $U$ o terço superior. Observe que $-1 \geq T \leq 1$.

Picos $(P)$ é uma característica que informa sobre padrões nos dados temporais que podem representar anomalias ou dados súbitos.

\begin{equation}
P = \sum_{i=1}^{N}{I(d_i > 2)}
\end{equation}

onde:

\begin{equation}
d_i = \frac{y_i - y_{T}}{s_{T}}
\end{equation}

sendo $y_{T}$ e $s_{T}$, respectivamente, a média e o desvio padrão aparados (i.e., após retirar 20\% dos dados da amostra, sendo 10\% referente aos menores valores e 10\% referente aos maiores valores).

A sazonalidade informa comportamentos que se repetem a cada ciclo (normalmente de 3 meses). \cite{BusingerRead1999} represetam a sazonalidade através da seguinte medida adimensional:

\begin{equation}
SS = 1 - \frac{ss_{w}}{ss_{T}}
\end{equation}

onde:

\begin{equation}
ss_{w} = \sum_{i=1}^{4}\sum_{j=i}^{n_i}{x_{ij}}
\end{equation}

e

\begin{equation}
ss_{T} = \sum_{i}\sum_{j}{(x_{ij}-\bar{x})^2}
\end{equation}

sendo:

\begin{equation}
\bar{x_i} = \frac{1}{n_i} \sum_{j=1}^{n_i}{x_{ij}}
\end{equation}

e

\begin{equation}
\bar{x} = \frac{1}{n} \sum_{i=1}^{4}{n_i \bar{x_i}}
\end{equation}

tal que: 

$y_{i}^w$ representa os dados winsorizados (i.e. após retirar 20\% dos dados da amostra, sendo 10\% referente aos menores valores e 10\% referente aos maiores valores, substituindo esses dados, respectivamente, pelo menor e pelo maior valor dentro do intervalo de dados restantes (\%80 dos dados)); 

$n=4k+r$ representa o número total de períodos para $r = 0, 1, 2, 3$, sendo o período $(n_i)$, é definido por:

\begin{equation}
n_i = 
\begin{cases}
4k, \ \quad \mathrm{se} \quad r =0 \\
4k + i, \quad \mathrm{se} \quad r > 0
\end{cases}
\end{equation}

\begin{equation}
x_{ij} = y_{i+4(j-1)}^w \quad i = 1,...,4 \quad j=1,...,k
\end{equation}

e

\begin{equation}
x_{i(k+1)} = y_{n_i}^w \quad i = 1,...,r \quad r>0
\end{equation}

De acordo com \cite{BusingerRead1999}, a característica corrida representa um intervalo no qual observações sucessivas ocorrem todas no mesmo lado da mediana do processo. \cite{BusingerRead1999} utilizam o número de corridas acima e abaixo da mediana. Os autores afirmam que um valor baixo para o número total de corridas pode confirmar uma tendência; enquanto que um valor alto sugere agrupamento ou outro comportamento oscilatório.

\begin{equation}
Run = \frac{|R - E[R]|}{\sqrt{Var(R)}}
\end{equation}

onde: 

R representa a soma dos números de corridas abaixo e acima da mediana (após exclusão dos valores iguais à mediana),

\begin{equation}
E[R] = 1 + \frac{2n_1n_2}{n_1 + n_2}
\end{equation}

\begin{equation}
Var(R) = \frac{2n_1n_2(2n_1n_2 - n_1 - n_2)}{(n_1 + n_2)^2(n_1+n_2-1)}
\end{equation}

sendo:

$n_1$ o número total de observação acima da mediana; e

$n_2$ o número total de observação abaixo da mediana.

A assimetria $(A)$ é a razão entre a média aritmética $(\bar{y})$ e a mediana $(m)$.

\begin{equation}
A = \frac{\bar{y}}{m}
\end{equation}

Valores de $A$ maiores que 1, indicam uma assimetria positiva, e valores de $A$ menores do que 1, indicam uma assimentria negativa.

A correlação $CR$ indica a relação entre dois semestres subsequentes:

\begin{equation}
CR = \frac{|RVN-2|}{\sigma}
\end{equation}

onde:

\begin{equation}
\sigma^2 = \frac{4(n-2)(5n^2-2n-9}{5n(n+1)(n-1)^2}
\end{equation}

\begin{equation}
RVN = \frac{12NM}{n(n^2-1)}
\end{equation}

\begin{equation}
NM = \sum_{i=1}^{n-1}{[r_i-r_{i+1}]^2}
\end{equation}

$r_i$ representa o rank de $y_i$.

Após calculadas, as características coeficiente de variação, número de zeros, tendência, picos (outliers), sazonalidade, corridas, assimetria e autocorrelação são utilizadas por \cite{BusingerRead1999} para agurpamento dos itens, sendo então identificado qual modelo ARIMA é mais adequado para cada grupo. 

\cite{Williams1984} apresenta um esquema para classificação da demanda em suave (smooth), de movimento lento, ou esporádica, particionando a variabilidade da demanda durante um lead time $(C_{LTD}^{2})$ em suas partes causais constituintes: variabilidade dos números de pedidos $(\frac{C_{n}^{2}}{\bar{L}})$, variabilidade dos tamanhos dos pedidos $(\frac{C_{x}^{2}}{\bar{n}\bar{L}})$ e variabilidade dos prazos de entrega $(C_{L}^{2})$. 

\begin{equation}
C_{LTD}^{2} = \frac{C_{n}^{2}}{\bar{L}} + \frac{C_{x}^{2}}{\bar{n}\bar{L}} + C_{L}^{2}
\end{equation}

onde:

$n$ representa os números de pedidos que chegam em unidades de tempo sucessivas (variáveis randômicas independetes e identicamente distribuídas (IIDRVs), com média $\bar{n}$ e variância $var(n)$), \\

$x$ representa os tamanhos dos pedidos (IIDRVs, com média $\bar{x}$ e variância $var(x)$),  \\

$L$ representa os prazos de entrega (IIDRVs, com média $\bar{L}$ e variância $var(L)$)), e \\

$C_{i}$ representa o coeficiente de variação de $i$. \\

Tal esquema foi posteriomente revisado por \cite{EavesKingsman2004}, considerando também o padrão de demanda irregular. A classificação adaptada por pelos autores é apresentada na tabela a seguir.

\begin{table}[h]
\begin{center}
\begin{tabular}[c]{c c c c}
\cline {1-4}
$\frac{C_{n}^{2}}{\bar{L}}$ & $\frac{C_{x}^{2}}{\bar{n}\bar{L}}$ & $C_{L}^{2}$ & padrão de demanda \\ \cline {1-4}
baixo & baixo &  & suave   \\ 
baixo & alto  &  & irregular   \\ 
alto  & baixo &  & de movimento lento   \\
alto  & alto  & baixo & intermitente   \\
alto  & alto  & alto  & atamente intermitente\\ \cline {1-4}
\end{tabular}
\label{tab:DemandPattern}
\caption{Classificação dos padrões de demanda de acordo com \cite{EavesKingsman2004}.}
\end{center}
\end{table}

Observe que os valores dos critérios de corte definidos para diferenciar alto e baixo são arbitrários.

\cite{SyntetosEtAl2005} sugerem um esquema de categorização contruído a partir da comparação do erro médio quadrado de três diferentes metodologias (método de Croston, método de Croton modificado e amortecimento exponecial simples). De acordo com esse esquema, os parâmetros quadrado do coeficiente de variação do tamanho da demanda $(CV^2)$ e intervalo médio entre demandas $(p)$ são usados para classificar a demanda entre errática, esporádica, suave e intermitente. A tabela a seguir apresenta a classificação proposta pelos autores.

\begin{table}[h]
\begin{center}
\begin{tabular}[c]{c c c}
\cline {1-3}
$CV^2$ & $p$ & \multirow{2}{*}{padrão de demanda} \\ 
$0.49$ & $1.32$ & \\ \cline {1-3}
baixo & baixo & suave   \\ 
baixo & alto  & intermitente   \\ 
alto  & baixo & errática   \\
alto  & alto  & esporádica  \\ \cline {1-3}
\end{tabular}
\label{tab:DemandPatternSybtetos}
\caption{Classificação dos padrões de demanda de acordo com \cite{SyntetosEtAl2005}.}
\end{center}
\end{table}

Observe que nesse esquema de classificação são definidos os pontos de corte $CV^2=0.49$ e $p=1.32$.

%\bibliographystyle{acm}
%\bibliography{Bibliography/BibFile}

\begin{thebibliography}{12}

\bibitem[\protect\citeauthoryear{Ballou}{2001}]{Ballou2001}
Ballou, R.H. (2001).{\it Gerenciamento da Cadeia de Suprimentos: Planejamento, Organização e Logística Empresarial}, 4. ed., Porto Alegre: Bookman.

\bibitem[\protect\citeauthoryear{Ballou}{2006}]{Ballou2006}
Ballou, R.H. (2006).{\it Gerenciamento da Cadeia de Suprimentos / Logística Empresarial}, 5. ed., Porto Alegre: Bookman.

\bibitem[\protect\citeauthoryear{Boylan et al.}{2008}]{BoylanEtAl2008}
Boylan, J.E., Syntetos, A.A., e Karakostas, G.C. (2008). 'Classification for forecasting and stock control:a case study'. {\it Journal of the Operational Research Society}, Vol. 59, pp. 473--481.

\bibitem[\protect\citeauthoryear{Businger e Read}{1999}]{BusingerRead1999}
Businger, M.P., e Read, R.R. (1999). 'Identification of demand patterns for selective processing: acase study'. {\it Omega, Int. J. Mgmt Sci.}, Vol. 27, pp. 189--200.

\bibitem[\protect\citeauthoryear{Eaves e Kingsman}{2004}]{EavesKingsman2004}
Eaves A.H.C., e Kingsman B.G. (2004). 'Forecasting for the ordering and stock-holding of spare parts'. {\it J. O. Opl. Res. Soc.}, Vol. 55, pp. 431--437.

\bibitem[\protect\citeauthoryear{Fraga}{2019}]{Fraga2019}
Fraga, T.B. (2019). 'Estudo de Métodos de Previsão de Demanda e Proposição de Metodologia Combinada no Contexto das Micro e Pequenas
Empresas do Agreste Pernambucano'. Projeto de Pesquisa registrado em 09/11/2019, e aprovado pela Pró-reitoria de Pesquisa da UFPE em 11/02/2021 (Processo SIPAC 23076.057489/2019-21).

\bibitem[\protect\citeauthoryear{Harvey e Sotardi}{2018}]{HarveySotardi2018}
Harvey, H.B., Sotardi, S.T. (2018). 'The Pareto Principle'. {\it Journal of the American College of Radiology}, Vol. 15 (6), pp. 931.
    
\bibitem[\protect\citeauthoryear{Makridakis et al.}{1998}]{MakridakisEtAl1998}
Makridakis, S.G.,Wheelwright, S.C., Hyndman, R.J. (1998).{\it Forecasting: Methods and Applications}, 3. ed., Wiley.

\bibitem[\protect\citeauthoryear{Makridakis e Hibon}{2000}]{MakridakisHibon2000}
Makridakis, S. e Hibon, M. (2000). 'The M3-Competition: results, conclusions and implications'. {\it International Journal of Forecasting}, Vol. 16, pp. 451--476.

\bibitem[\protect\citeauthoryear{Syntetos et al.}{2005}]{SyntetosEtAl2005}
Syntetos, A.A., Boylan, J.E., e Croston, J.D. (2005). 'On the categorization of demand patterns'. {\it Journal of the Operational Research Society}, Vol. 56 (5), pp. 495--503.

\bibitem[\protect\citeauthoryear{Ulrich et al.}{2022}]{UlrichEtAl2022}
Ulrich, M., Jahnke, H., Langrock, R., Pesch, R., e Senge, R. (2022). 'Classification-based model selection in retail demand forecasting'. {\it International Journal of Forecasting}, Vol. 38 (1), pp. 209--223.

\bibitem[\protect\citeauthoryear{Werner et al.}{2006}]{WernerEtAl2006}
Werner, L, Lemos, F.O., Daudt, T. (2006). 'Previsão de demanda e níveis de estoque uma abordagem conjunta aplicada no setor siderúrgico'. {\it XIII SIMPEP}, Bauru, SP, Brasil.

\bibitem[\protect\citeauthoryear{Williams}{1984}]{Williams1984}
Williams, T.M. (1984). 'Stock control with sporadic and slow-moving demand'. {\it Journal of the Operational Research Society}, Vol. 35 (10), pp. 939–948. 

\end{thebibliography}

\end{document}
